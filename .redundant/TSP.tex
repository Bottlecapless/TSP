\documentclass[12pt]{article}  % 官方要求字号不小于 12 号, 此处选择 12 号字体
 \linespread{1.1} %用来调整行距, 为标准行距的1.1倍
\usepackage{geometry} %此宏包用来调整页边距
    \geometry{left=1in,right=0.75in,top=1in,bottom=0.8in} % 搭配geometry宏包食用, 将页边距调整至特定格式.
%\usepackage{mathptmx}  % 这是 Times 字体, 中规中矩 
\usepackage{palatino}  % palatino是 COMAP 官方杂志采用的更好看的 Palatino 字体, 可替代以上的 mathptmx 宏包. 在正文中可以使用: 无衬线字体、等宽字体、斜体字体、粗体字体. 
\usepackage{pdfpages} % 可以插入一页pdf, 注意只能一页. 
\usepackage{longtable} % 可以插入跨页长表格, 同时支持冻结首行功能, 即跨页后表格第一行仍然为原表格第一行. 
\usepackage{tabu} % 更加灵活的表格制作, 可以设置列宽、行高、单元格格式, 我常用\rowfont[alignment]{fontSpec}把某行的内容置为粗体.
\usepackage{threeparttable} % 允许表格底部增加表格注释
\usepackage{listings} % 提供代码环境
\usepackage{paralist} % 提供更加方便的列表环境, 包括创建紧凑的列表、创建带有括号的行内列表
\usepackage{booktabs} % 创建更漂亮的表格, 比如经典的三线表
\usepackage{newtxtext} % 提供更多的字体
\usepackage{amsmath,amssymb,amsthm} % 经典, 没它们我会死
\usepackage{newtxmath} % must come after amsXXX
\usepackage{graphicx} % 允许插入图片、位图等. 
\usepackage{xcolor} % 允许对颜色做处理
\usepackage{fancyhdr} % 更灵活的页眉和页脚,包括自由的页码
\usepackage{float} % 为浮动体提供H命令
\usepackage{lastpage}
\usepackage{hyperref} % 允许我进行超链接
\hypersetup{colorlinks=true,urlcolor=blue}
\hypersetup{colorlinks=true,linkcolor=red}
\hypersetup{colorlinks=true,citecolor=blue}
\usepackage{subfigure} % 子图
\usepackage{multirow} % 合并列单元格

\usepackage{algorithm}% 为算法提供浮动环境
\usepackage{algpseudocode}
\usepackage{tikz}
\usetikzlibrary{arrows.meta, positioning}

\usepackage[AutoFakeBold, AutoFakeSlant]{xeCJK}

\newtheorem{theorem}{Theorem} 
\newtheorem{corollary}{Corollary}
\newtheorem{lemma}{Lemma}
\newtheorem{definition}{Definition}
\newtheorem{proposition}{Proposition} % 呃呃太熟悉了这些

\newcommand{\blue}[1]{{\color{blue}#1}}
\renewcommand{\bf}[1]{\textbf{#1}}
% \newcommand{\bbf}[1]{{\textbf{\color{blue}#1}}}

\title{{TSP}}

\author{王英枭 \\
          211840213\\
         {\small wangbottlecap@gmail.com}}

% \bibstyle{alpha}
\bibliographystyle{alpha}

\lstset{
  language=Python,
  basicstyle=\ttfamily\small,
  numbers=left,
  numberstyle=\tiny\color{gray},
  stepnumber=1,
  numbersep=5pt,
  backgroundcolor=\color{white},
  showspaces=false,
  showstringspaces=false,
  showtabs=false,
  frame=single,
  rulecolor=\color{black},
  tabsize=2,
  captionpos=b,
  breaklines=true,
  breakatwhitespace=false,
  title=\lstname,
  keywordstyle=\color{blue},
  commentstyle=\color{green},
  stringstyle=\color{red},
  escapeinside={\%*}{*)},
  aboveskip = 10pt
  % belowskip = 10pt
}

\begin{document}

\maketitle

\begin{abstract}
离散优化课程作业,主要包括求解器Gurobi、COPT对旅行商问题样例d657, rat575, u1060的实现。对两种求解器的调用均放弃MTZ子环约束,转而使用Callback Module进行Lazy Constraint限制。此外,添加了最近邻算法以加速寻找初始解,并在适当的地方进行求解器参数调优。
\end{abstract}

\tableofcontents
\section{数学模型}
定义两个城市$i, j$间的距离为其二维坐标下的欧几里得距离,目标函数因此被定义为路径$\omega$在此城市距离定义下的总长度。物理约束仅添加``每个城市$i$的度(degree)为2'',子环在Callback中考虑。

\section{Gurobi}
使用Gurobi进行计算的配置如下:
\begin{enumerate}
    \item Callback: 只有找到合法子环,才添加lazy constraint把该子环排除;放弃在硬约束中排除所有子环
    \item 求最短子环的函数不会找到对称子环,以破解对称性
    \item 在Gurobi求解之前使用最近邻算法生成初始可行路径,并设置为初始解
    \item 参数调优:
    \begin{itemize}
        \item model.setParam("MIPFocus", 1)
        \item model.setParam("ImproveStartTime", 5)
        \item model.setParam("ImproveStartNodes", 100)
        \item model.setParam("Cuts", 1)
        \item model.setParam("Heuristics", 0.8)
        \item model.setParam("Presolve", 1)
        \item model.setParam("Threads", 8)
    \end{itemize}
\end{enumerate}
用于数值比较的参数配置(如计算时间=timeLimit,绝对最优间隙=0.000001等)已略去。

\section{COPT}
使用COPT进行计算的配置如下:
\begin{enumerate}
    \item Callback: 只有找到合法子环,才添加lazy constraint把该子环排除;放弃在硬约束中排除所有子环
    \item 求最短子环的函数不会找到对称子环,以破解对称性
    \item 在COPT求解之前使用最近邻算法生成初始可行路径,并设置为初始解
    \item 参数调优:
    \begin{itemize}
        \item model.setParam(COPT.Param.HeurLevel, 3)
        \item model.setParam(COPT.Param.Threads, 8)
        \item model.setParam(COPT.Param.Presolve, 2)
    \end{itemize}
\end{enumerate}
用于数值比较的参数配置(如计算时间=timeLimit,绝对最优间隙=0.000001等)已略去。

\section{代码公开}
代码已在Github公开,具体细节参见\url{https://github.com/Bottlecapless/TSP.git},谢谢!

\end{document}